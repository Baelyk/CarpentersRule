\documentclass{article}

\usepackage[utf8]{inputenc}
\usepackage{amsmath,amsfonts,fullpage,amsthm,amssymb}
\usepackage{thmtools}
\usepackage{enumitem}

% Better matrices, namely augmented
% https://tex.stackexchange.com/a/2244
\makeatletter
\renewcommand*\env@matrix[1][*\c@MaxMatrixCols{c}]{%
	\hskip -\arraycolsep
	\let\@ifnextchar\new@ifnextchar
	\array{#1}}
\makeatother

\usepackage{natbib}
\usepackage{graphicx}
\usepackage{ifthen}
% Code blocks
\usepackage[outputdir=latex-build]{minted}

\usepackage[table]{xcolor}

% TikZ for Graphs
\usepackage{tikz}
\usepackage{tikz-3dplot}
\usetikzlibrary{positioning, quotes}

% Defining shortcut macros
\newcommand\Z{\mathbb{Z}}
\newcommand\R{{\mathbb{R}}}
\newcommand\Q{{\mathbb{Q}}}
\newcommand\C{{\mathbb{C}}}
\newcommand\N{{\mathbb{N}}}
\newcommand\F{{\mathcal{F}}}
\newcommand\TR{{\operatorname{tr}}}
% Replace \P make paragraph symbol with \P makes \mathcal{P} for power sets
\renewcommand\P{{\mathcal{P}}}
\newcommand{\then}{\Rightarrow}

\DeclareMathOperator{\im}{Im}
\DeclareMathOperator{\arccot}{arccot}

\declaretheoremstyle[
	spaceabove={1em plus 0.1em minus 0.2em},
	notefont=\bfseries,
	notebraces={}{},
	headformat=\ifthenelse{\equal{\NOTE}{}}{\NAME{} \NUMBER}{\!\!\NOTE}
]
{thm}
\usetikzlibrary{quotes,angles}
\begin{document}

\begin{figure}[h]
	\centering
	\begin{tikzpicture}
		\draw [black, thick](0, 0)-- (6, 0);
		\draw[black, thick] (0,0)-- (4, 4) ;
	    \filldraw[red] (0,0) circle (3pt) node[anchor=west]{};
	    \filldraw[red] (4.242640687,0) circle (3pt) node[anchor=west]{};
	    \filldraw[red] (3,3) circle (3pt) node[anchor=west]{};
		\draw
        (3,3) coordinate (x) node[anchor = north west] {\Huge x}
        -- (0,0) coordinate (v) node[left] {\Huge v}
        -- (4.242640687,0) coordinate (y) node[below] {\Huge y}
        pic["\Huge$ \theta$",draw=blue, very thick, <->, angle eccentricity=1.5, angle radius=1cm] {angle=y--v--x};
		
		
	\end{tikzpicture}
	% \caption{A vertex $v$ with a angle of $\theta$ and points $x$ and $y$ equidistant to $v$}
	\label{fig:adjacent chordarc}
\end{figure}

\end{document}

		